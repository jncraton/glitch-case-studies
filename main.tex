\documentclass[letterpaper]{article}
\usepackage{graphicx} % Required for inserting images
\usepackage{authblk}
%configure print geometry
\usepackage[margin=1.2in]{geometry}
\setlength{\parskip}{0.3\baselineskip}
% preferred reference package
\usepackage[style=numeric, sorting=none]{biblatex}
\addbibresource{refs.bib}

\title{Classic Game Glitches as Case Studies in Cybersecurity and Computer Science Classrooms}
\author[1]{Jon Craton}
\affil[1]{Anderson University, Anderson, IN}
\date{} % no date, we will add it to the final version in the header

\begin{document}

\maketitle

\begin{abstract}
This work proposes using classic video game glitches as case studies to demonstrate key concepts in computer science and cybersecurity. An example case is provided and explored. The case is evaluated as a learning instrument as part of its use in an undergraduate computer science classroom.
\end{abstract}

\section{Introduction}
One challenging aspect of teaching is presenting material in a manner that is both authentic and engaging. While it would be nice if we could all arrive to a learning experience prepared to learn for the sake of learning, this is not always the case. While it is certainly possible to coerce students into educational activities via grades or the threat of a quiz or exam, it is beneficial to find ways to encourage intrinsic motivation \cite{deci2013intrinsic}.

As educators, it can be tempting to lean heavily on direct instruction. It allows for a relatively simple planning and execution while making a teacher out of anyone who can read text off of a slide deck. However, a large body of research demonstrates that this approach is not best for learners. Active learning enhances student performance in STEM fields \cite{freeman2014active} and helps to close achievement gaps for underrepresented students \cite{theobald2020active}.

Case studies are a common tool in education. They allow students to explore and apply concepts to real-world situations. They have been shown to be valuable in computer science education as they may be better able to capture the multi-dimensional decisions necessary to construct a correct and effective computer program \cite{linn1992case}.

Video game glitches are engaging to some learners by their nature. It has been noted that some people play video games enjoy video game glitches: "one of the many ways in which a player may enjoy a video game is to explore it in search of these discrepancies, relishing the variety of effects they can cause" \cite{bainbridge2007creative}.

\section{Theoretical Foundation}
It can be intimidating for educators to include live demonstrations in their teaching, especially if these demonstrations require skill and can fail in various ways for technical reasons. A lecture has very few failure modes, whereas more complex learning experiences can run into various challenges. The benefits of taking this risk are well-documented. As Ken Bain notes: "The best teaching is often both an intellectual creation and a performing art. It is both Rembrandt’s brush strokes and the genius of insight, perspective, originality, comprehension, and empathy that makes a Dutch Master." \cite{bain2004best}.

In order for teaching to be engaging for students, it is important for it to be genuinely exciting for the teacher. As Parker Palmer notes, "As I teach I project the condition of my soul onto my students, my subject, and our way of being together." \cite{palmer2000courage}. Finding ways to add real exploration and excitement to teaching is valuable both for both teachers and learners.

Part of good teaching is creating learning experiences that allow students to become open to a state where the deepest forms of knowledge acquisition can occur. It is not enough for students to listen and take notes, they must at times be truly confused and confront their ignorance with humility. This state has been described by Csikszentmihalyi in various ways including  "what a painter feels when the colors on the canvas begin to set up a magnetic tension with each other, and a new thing, a living form, takes shape in front of the astonished creator" \cite{csikszentmihalyi1990flow}.

Students are not merely learning static concepts that can be conveyed as facts from one person to another. Situated learning reminds us that a concept "will continually evolve with each new occasion of use, because new situations, negotiations, and activities inevitably recast it in a new, more densely textured form" \cite{brown1989situated}.

Kolb identified that true learning happens when we encounter a new experience, reflect on it, develop an abstract conceptualization, and engage in active experimentation \cite{kolb84}. Glitches in classic games provide an ideal platform to craft this learning cycle. An initial exploration brings with it a moment of confusion as assumptions about the simulated world are violated. This is followed by additional observation of the world that allows learners to construct a more accurate model of what is really happening behind the scenes. Finally, learners can test assumptions about their understanding via further experimentation in the world.

\section{Methodology for Classroom Demonstration}

An exploration of an in-game glitch first requires a mechanism that allows the glitch to be demonstrated and explored live in a classroom setting. This can be done by connecting a classic console to the classroom projection system, but this creates a number of challenges. It is tedious to transport and connect the system, but it also has limited ability to inspect and debug running games. For this reason, it is often best to use some form of emulation. For this exploration, the open source higan emulator was used \cite{ginder2004higan}.

In order to run a game under emulation, it is necessary to create an image of the read-only memory found on a game cartridge. There are various hardware tools and methods to accomplish this. Once the ROM file has been copied it can be stored for later use. Specialized hardware is not required during emulation.

One desirable aspect of emulation technology is the ability to pause, slow down, or rewind time. Many game glitches require time consuming setup or precise execution, and having the ability to retry an attempt or quickly experiment with alternative approaches in response to live student feedback is desireable. Many emulation systems provide the ability to create and restore the entire state of system at a given moment. This feature can be used to rapidly jump to a desired point within a game that would otherwise consume significant class time to discover.

\section{Example Case}
The case proposed in this paper involves a well-known glitch colloquially referred to as the "Old man glitch". It is well understood and its underlying mechanics have already been described \cite{bulbapedia2005} \cite{scrumpy2016missing}.

This glitch induces the system to access and use memory that has not been properly initialized. In particular, it requires the player character in the game to engage in a wild Pokemon battle when the wild Pokemon data is not properly initialized. This allows the player to encounter and capture invalid or "glitch" Pokemon as shown below:

\noindent % Prevents indentation
\begin{minipage}{\textwidth}
    \centering
    \includegraphics[width=0.4\textwidth]{missingno.png}
    \label{fig:missingno}
\end{minipage}

As a the player character navigates the virtual work, they have opportunities to discover and battle wild Pokemon. The Pokemon faced are drawn from list that is populated by game region, so that players are presented with access to new varieties as they progress through areas of the game.

At one point in the game, the player is taught how to catch Pokemon by a character known simply as "Old Man". In order to demonstrate catching Pokemon, the game scripts what would normally be player actions in order to demo catching a Pokemon as a cutscene. A naive approach to this would leave the player's name as the one engaging in the encounter, so the players name must be temporarily replace by "Old Man" in memory. The software later needs to restore the player's correct name, so it must be stored elsewhere temporarily. The memory location used for temporary storage happens to be the location for some wild Pokemon encounter data, as it is not needed in the region where "Old Man" is found.

After the demonstration cutscene, the player's name is restored, but the wild Pokemon encounter data is left in an invalid, uninitialized state. This data is not needed in the current region, and it is re-initialized as the player moves regions, so it does not generally cause an issue. However, if the player can move to a region that does not reinitialize this part of the wild Pokemon encounter data and trigger an appropriate wild Pokemon encounter, improperly initialized data will be used to select a Pokemon for the encounter. This can be exploited by immediately flying to Cinnabar Island.

Cinnabar Island was not intended to include wild Pokemon encounters, so it does not re-initialize the wild Pokemon data. Data from the previous region, or in this case improperly initialized data, would be used for any encounters. Due to a separate bug, one small strip of water to the east of Cinnabar Island does allow encounters, and these will trigger the glitch.

\noindent % Prevents indentation
\begin{minipage}{\textwidth}
    \centering
    \includegraphics[width=0.4\textwidth]{surfing.png}
    \label{fig:surfing}
\end{minipage}

This glitch study provides numerous opportunities for learning. A real interactive example like this is multifaceted and can be explored in real time. There are several issues that work together to allow the final issue, showing the importance of defense in depth.

From a cybersecurity standpoint, it can be valuable to consider the root cause of this issue. Perhaps it could be mapped to an appropriate framework, such as identifying the cause as "Use of Uninitialized Resources" (CWE-908) \cite{mitre2012}. From there, learners can discuss and explore potential mitigations for this sort of software weakness.

It is also possible take this example farther if desired. For example, what happens when we catch one of these glitch Pokemon? Do these encounters have any impact on the rest of the game? As it turns out, encounters with these invalid Pokemon trigger an invalid write and cause the quantity of the sixth item in a player's inventory to be set to 128 \cite{bulbapedia2010}. This can be used to demonstrate how seemingly benign software issues can sometimes to exploited to gain elevated privileges in unexpected ways.

\section{Assessment}
Students were pre-tested prior to the learning experience and post-tested afterwards. Quantitatively, a paired-samples t-test indicated a statistically significant increase in scores from the pre-test (M=70.37,SD=30.93) to the post-test (M=92.59,SD=14.70), with a p-value of p=0.0497.

Qualitatively, students reported being more actively connected to their learning experience via the case study. One student called it "very engaging", and another referred to it as being "much better than usual teaching".

This assessment is admittedly limited. It measures the impact of a single case in a single relatively small classroom, but it does demonstrate that their may be potential to this type case study tool as a way to enhance student engagement and learning.

\secion{Discussion and Future Work}

The idea of video game gliches as case studies of software engineering, computer science, and cybersecurity is compelling for many students. It allows them to connect experiences from their lives with content from the classroom and provides a mechanism for situating classroom knowledge within real software systems.

Video games provide a very natural presentation for software concepts. By their nature, they are not merely visible but also often flashy and attention grabbing. By violating the rules of simulated environments, educators can create moment of confusion and delight that capture attention and provide a platform for diagnosing and exploiting complex software issues.

While this study is limited in scope and explores only a single case in a single classroom, this work could be significantly expanded in the future. Additional cases could be crafted and shared for use in a broader set of classrooms. Cases could also be evaluated more rigorously and broadly by expanding studies to additonal classrooms and comparing this case study approach against other teaching methods that have similar aims.

%\bibliographystyle{plain}
%\bibliography{refs}

\printbibliography
\end{document}
